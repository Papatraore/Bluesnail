\documentclass[11pt]{article}

% ENCODAGE CARACTERES
\usepackage[utf8]{inputenc}
\usepackage[T1]{fontenc}
\usepackage[frenchb]{babel}

% MISE EN FORME
\usepackage[left=3cm,right=3cm,top=3cm,bottom=3cm]{geometry}
\usepackage[colorlinks=true,urlcolor=blue]{hyperref}
\usepackage{xcolor}
\usepackage{enumitem}
\usepackage{pifont}
\usepackage{placeins}
\usepackage{listings}
\usepackage{color}

\hypersetup{linkcolor=blue}

% Pour les listes à puces
\frenchbsetup{StandardLists=true} 
\setitemize[0]{font=\color{black},label=\ding{227},itemsep=0pt}

\title{BlueSnail - Documentation}
\author{Alexis Bonnin - Salma ... (écrivez vos noms)}
\date{Avril 2017}

\begin{document}

\maketitle

\begin{abstract}
    Cette documentation s'adresse d'une part aux utilisateurs souhaitant utiliser l'application \textit{roger-runner} et d'autre part, aux développeurs qui veulent créer de nouveaux plugin pour cette application. Vous trouverez également dans cette documentation une description de la plate-forme \textit{BlueSnail}.
\end{abstract}

\tableofcontents

\newpage

\section{Introduction}

\subsection{Présentation du projet}
    L'application \textit{roger-runner} est un jeu de d'arcade fonctionnant via la plate-forme \textit{BlueSnail}. Ce jeu consiste à déplacer un personnage tout en évitant les différents obstacles. Il se joue au clavier et plus précisément grâce aux touches directionnelles ainsi que la touche "barre espace".
    
    Cette documentation décrit comment installer la plate-forme et les différents plugin et comment développer un nouveau plugin. Elle présente également une description de la plate-forme \textit{BlueSnail}.
    
    Pour fonctionner, ce projet nécessite :
    \begin{itemize}
        \item Linux
        \item Java (version 1.7 ou plus)
        \item Maven (version 3)
    \end{itemize}
    
\subsection{Vocabulaire}
\label{subsec:vocabulaire}

    Pour éviter toute confusion, voici quelques notions importantes à savoir :
    \begin{itemize}
        \item \textbf{plate-forme} : La plate-forme est le point d'entrée du programme. Celle-ci est chargée de gérer les différents plugin ainsi que de lancer ceux en mode "autorun". Tout besoin en terme d'extension doit être formulé à cette plate-forme.
        
        \item \textbf{application} : On distingue deux types de plugin : les \textit{applications} et les \textit{extensions}. Les applications sont des plugin en mode "autorun", c'est-à-dire qu'ils ne dépendent pas de d'autre plugin et sont exécutés dès le démarrage de la plate-forme.
        
        \item \textbf{extension} : les extensions sont des plugin dépendant d'une application. Ils ne seront exécutés par la plate-forme que lorsqu'un besoin sera formulé par une application.   
    \end{itemize}

\section{Installer et utiliser l'application}

\subsection{Installation}
    Tout d'abord, le projet est disponible sur GitHub à l'adresse suivante :  \url{https://github.com/Asteip/Bluesnail.git}. Pour installer le projet, veuillez exécuter les commandes suivantes dans un terminal :
    
    \begin{itemize}
        \item Cloner le dépôt :
        \begin{lstlisting}[language=bash]
        git clone https://github.com/Asteip/Bluesnail.git
        \end{lstlisting}
        
        \item Installer la plate-forme ainsi que tous les plugin (à la racine du projet) :
        \begin{lstlisting}[language=bash]
        ./bluesnail.sh -i
        \end{lstlisting}
        
        \item Afficher l'aide :
        \begin{lstlisting}[language=bash]
        ./bluesnail.sh -h
        \end{lstlisting}
    \end{itemize}
    
\subsection{Désinstallation}
    Pour désinstaller l'application, il suffit de supprimer le répertoire "Bluesnail" contenant le projet.
    
\subsection{Lancer le programme}
    Pour exécuter l'application, veuillez lancer la commande qui suit dans un terminal :

    \begin{itemize}
        \item Exécuter les plugin autorun :
        \begin{lstlisting}[language=bash]
        ./bluesnail.sh -r
        \end{lstlisting}
    \end{itemize}

    Ceci aura pour effet de lancer toutes les applications disponibles, c'est-à-dire tous les plugin en mode "autorun". Si une application, une extension ou la plate-forme n'est pas installée alors celle-ci sera automatiquement installée avant l'exécution.
    
\subsection{Comment jouer à Roger-runner ?}
    \paragraph{But du jeu}
    L'objectif de Roger-runner est d'éviter les aliens apparaissant sur la carte. Le jeu se termine si Roger se fait tuer par un alien.
    
    \paragraph{Commandes}
    Tout d'abord, pour déplacer Roger, il faut utiliser les touches directionnelles : les flèches gauche et droite permettent de déplacer le personnage à gauche et à droite de l'écran. La flèche du haut permet de sauter. Enfin, la touche "barre espace" permet d'envoyer des projectiles sur les ennemis.
    
    
\section{Développer un plugin}

    \noindent Tout d'abord, le projet est organisé en cinq répertoires :
    \begin{itemize}
        \item \textbf{platform} : Contient le code source ainsi que les fichiers compilés de la plate-forme.
        
        \item \textbf{applications} : Contient les différentes applications.
        
        \item \textbf{extensions} : Contient les différentes extensions.
        
        \item \textbf{documentation} : Contient la documentation technique ainsi que la documentation du code source (javadoc).
        
        \item \textbf{template} : Contient des templates utilisés pour la création de plugin.
    \end{itemize}

\subsection{Créer un nouveau plugin}
\label{subsec:creation_plugin}

    Il est possible de créer deux types de plugin : application ou extension (voir \ref{subsec:vocabulaire}). Pour créer un plugin, exécuter une des commandes suivantes :
    
    \begin{itemize}
        \item Créer une application :
        \begin{lstlisting}[language=bash]
        ./bluesnail.sh -c <app-name>
        \end{lstlisting}
        Où <app-name> correspond au nom de l'application à créer.
        
        \item Créer une extension :
        \begin{lstlisting}[language=bash]
        ./bluesnail.sh -c <extension-name> -p <app-name>
        \end{lstlisting}
        Où <extension-name> correspond au nom de la nouvelle extension et <app-name> correspond à l'application à étendre.
    \end{itemize}
    
    Une fois le plugin créé, un répertoire portant son nom et contenant le projet maven est créé, soit dans le répertoire "applications" si le plugin est une application, soit dans le répertoire "extensions" si le plugin est une extension. Pour importer un projet maven dans eclipse :
    
    \begin{enumerate}
        \item Ouvrir la version eclipse située dans \textit{/media/enseignants/...}
        \item Sélectionner "file" $\rightarrow$ "import" $\rightarrow$ "Maven" $\rightarrow$ "Existing Maven Projects" puis "next"
        \item Choisir le répertoire contenant le pom.xml (racine du projet)
        \item Cliquer sur "finish"
    \end{enumerate}

\subsection{Configurer un plugin}

    Lors de la création d'un plugin, celui-ci est configuré "par défaut" et contient les éléments suivants : 
    
    \begin{itemize}
        \item Un fichier pom.xml configuré
        \item Un package "com.alma.application" ou "com.alma.extension" suivant le type de plugin.
        \item Une classe générée automatiquement.
    \end{itemize}
    
    Le projet contient également un répertoire pour les tests si vous souhaitez tester le plugin avant la mise en service.
    
    \paragraph{\textcolor{red}{Important}}
    \textcolor{red}{Le nom du package principal ne doit pas être modifié (il est possible de créer d'autre package en cas de besoin). De plus, chaque classe créée (quelque soit le projet) doit avoir un nom unique.}
    
    \paragraph{Configurer la classe principale d'une application} 
    Si le plugin créé est une application, alors la classe principale doit impérativement implémenter l'interface "IMainPlugin" de la plate-forme (Note : la classe à importer est "com.alma.platform.data.IMainPlugin" ; les dépendances sont déjà configurées dans le fichier pom.xml). L'interface "IMainPlugin" impose de redéfinir la méthode run(), cette méthode est automatiquement appelée lors de l'exécution de l'application.
    
    \paragraph{Configurer la classe principale d'une extension} 
    Si le plugin créé est une extension d'une application existante, alors la classe principale de l'extension doit implémenter l'interface correspondant au besoin réalisé (Note : la classe à importer est de la forme "com.alma.application.<interface>" où "interface" correspond au besoin à réaliser ; les dépendances vers l'application son déjà configurées, d'où l'importance de ne pas modifier le nom du package principal).
    
    \paragraph{Configurer la classe principale d'un Monitor} 
    Si le plugin créé est un monitor, alors non seulement il doit implémenter l'interface "IMainPlugin", mais il doit aussi implémenter "IMonitor". De plus, pour être pris en compte par la plate-forme, le monitor doit s'ajouter à la liste des monitor (addMonitor(IMonitor monitor)).
    
    \paragraph{Fichier de configuration} Le fichier "config.txt" contient les informations nécessaires au chargement des plugin dans la plate-forme. Il est situé à la racine du répertoire "platform". Lors de la création d'un plugin, une ligne est ajoutée à ce fichier contenant des règles préétablies. Celle-ci doit être modifiée afin de répondre au mieux au besoin de l'utilisateur. La syntaxe suivante doit être respectée :
    
    \begin{itemize}
        \item Un plugin est déclaré sur une seule ligne
        
        \item La syntaxe d'une ligne est la suivante (présentée ici sur plusieurs ligne pour une meilleure lisibilité) :
        \begin{lstlisting}[language=Bash,extendedchars=true]
        name=<nom du plugin>;
        class=com.alma.extension.<nom classe principale>;
        interface=<nom interface implementee>;
        directory=extensions/<nom application>/target/classes/;
        autorun=\{true|false\}
        \end{lstlisting}
        Le nom de la classe principale correspond au nom de la classe qui va être chargée en première (ie : celle qui implémente l'interface correspondant au besoin) ; le nom de l'interface implémentée correspond au besoin réalisé.
    \end{itemize}
    
\subsection{Utiliser l'API de la plate-forme}

    La plate-forme (cf. section \ref{sec:desc_plateforme}) offre plusieurs méthodes utilisables par les plugin. Ces méthodes sont détaillées dans la java-doc de la plate-forme (dossier documentation/doc-platform/). Les principales méthodes de l'API sont :
    
    \begin{itemize}
        \item getInstance() : Retourner l'instance unique de la plate-forme.
        \item getListPlugin() : Retourne la liste de tous les plugin disponibles.
        \item getListPlugin(java.lang.Class<?> need) : Retourne la liste des plugin disponibles correspondant au besoin (need). Le besoin est exprimé via une interface que doit implémenter la classe principale du plugin.
        \item getPluginInstance(PluginDescriptor plugin) : Retourne une instance du plugin passé en paramètre.
    \end{itemize}

\subsection{Compiler un plugin}
    Pour compiler un plugin, il existe trois possibilités :
    \begin{itemize}
        \item Sur eclipse : si le projet a été importé sous eclipse, il est possible de compiler directement l'intégralité des projets (le projet de la plate-forme ainsi que les autres plugin doivent être ouvert).
        \item Avec maven : pour compiler le plugin en ligne de commande, il suffit de se déplacer dans le répertoire du plugin et de taper "mvn package".
        \item Avec le script bash : la commande "./bluesnail -i" aura pour effet de recompiler tous les plugin existants ainsi que la plate-forme.
    \end{itemize}
\section{Description de la plate-forme}
\label{sec:desc_plateforme}

\subsection{Structure de la plate-forme}




%une documentation décrivant  la plateforme, les choix de conception et les fonctionnalités qu'elle offre, sa configuration éventuelle... Ce document doit faciliter l'évaluation et la maintenance de la plateforme.

%Les informations indiquant comment votre application illustre l'utilisation des fonctionnalités de votre plateforme peuvent être placées dans la documentation de votre choix.

\end{document}
